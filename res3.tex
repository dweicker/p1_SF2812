This section presents the results for the third model (stochastic programming). It sometimes refers to the mathematical model available in the modelisation section.

The first thing to do is to define the different scenarios. Our group chose three different scenarios, called low, medium and high. The low scenario is when all the actual capacities are below the mean by a certain factor $p$. The medium scenario is when the real values are actually the one given. And the high scenario is when the values are above the mean by a factor $p$. So if we reuse the notation defined above :

$$c_{i,j,s} = \left\lbrace \begin{array}{c}
(1-p)c_{i,j} \quad\quad \text{if $s=1$}\\ 
c_{i,j} \quad\quad\qquad\:\:\:\: \text{if $s=2$}\\ 
(1+p)c_{i,j} \quad\quad \text{if $s=3$}
\end{array}\right.$$ 