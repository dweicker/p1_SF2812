\subsection{Results for stochastic model}
This section presents the results for the third model (stochastic programming). It sometimes refers to the mathematical model available in the modelisation section.

The first thing to do is to define the different scenarios. Our group chose three different scenarios, called low, medium and high. The low scenario is when all the actual capacities are below the mean by a certain factor $p$. The medium scenario is when the real values are actually the one given. And the high scenario is when the values are above the mean by a factor $p$. So if we reuse the notation defined above :

$$c_{i,j,s} = \left\lbrace \begin{array}{c}
(1-p)c_{i,j} \quad\quad \text{if $s=1$}\\ 
c_{i,j} \quad\quad\qquad\:\:\:\: \text{if $s=2$}\\ 
(1+p)c_{i,j} \quad\quad \text{if $s=3$}
\end{array}\right.$$ 

We also need to assign probability to each scenario. To preserve the fact that the mean value should be $c_{i,j}$, we have that $p_1=p_3$. We chose :
$$p_{s} = \left\lbrace \begin{array}{c}
0.25 \quad\quad \text{if $s=1$}\\ 
0.5 \qquad\:\: \text{if $s=2$}\\ 
0.25 \quad\quad \text{if $s=3$}
\end{array}\right.$$ 

With those scenarios and a chosen factor of $p=0.1$, we get the following expected utility : 
$$\sum_{s=1}^3 p_s\: maxUtility_s = 0.8615$$

The result is a little bit over the utility obtained when we have certainty over the capacities. It is to be expected. Having complete information will yield better results. We also present below the utilities for each scenario :
$$ 
\begin{tabular}{|c|c|c|c|}
\hline 
 & $s=1$ & $s=2$ & $s=3$ \\ 
\hline 
$maxUtility_s$ & 0.952 & 0.857 & 0.779 \\ 
\hline 
\end{tabular} $$

We can see that, the better the scenario, the better the utility. That is quite intuitive. We can also note that if $s=2$ (the capacities are the same as in the deterministic model), then it is possible to route the traffic to find the optimal value found with the deterministic model.

Let us finally look at the proposed traffic. The 40 Gbit/s-flow is given in the table below. In our model, this corresponds to the variables $x_{i,j}$.
\begin{center}
\begin{tabular}{|c|c|c|c|c|c|}
\hline 
i\textbackslash j & Sto & Par & Ber & War & Mad \\ 
\hline 
Sto &  &  &  & 8.762 &  \\ 
\hline 
Lon & 8.762 & 17.333 &  &  & 13.905 \\ 
\hline 
Par &  &  & 17.333 &  &  \\ 
\hline 
Ber &  &  &  & 17.333 &  \\ 
\hline 
Mad &  &  &  & 13.905 &  \\ 
\hline 
\end{tabular} 
\end{center}

Note that this traffic is fixed and cannot depend on the scenarios. We can also check that we have indeed 40Gbit/s leaving Warsaw and 40 Gbit/s entering London. The conservation of flow is also respected. 

Let us compare this solution to the one obtained for the deterministic model. We can see that it does not change much. 

Let us also look at the 50Gbit/s-flow. In our model, this is the variable $y$. It can depend on the scenarios so we will have a different flow for each scenario. The table below gives the results.

\begin{center}
\begin{tabular}{|c|c|c|c|}
\hline 
 & s = 1 & s = 2 & s = 3 \\ 
\hline 
Sto - Lon & 0.667 & 0.667 & 0.667 \\ 
\hline 
Sto - Par & 12 & 12 & 12 \\ 
\hline 
Sto - Ber & 21.429 & 20.381 & 20.381 \\ 
\hline 
Sto - War & 15.905 & 16.952 & 16.952 \\ 
\hline 
Lon - Par & 0.667 & 0.667 &  \\ 
\hline 
Lon - Mad &  &  & 0.667 \\ 
\hline 
Par - Mad &  & 11.333 &  \\ 
\hline 
Par - Rom & 14.190 & 2.857 & 13.524 \\ 
\hline 
Ber - Par & 1.524 & 1.524 & 1.524 \\ 
\hline 
Ber - War & 4.476 & 3.429 & 3.429 \\ 
\hline 
Ber - Rom & 15.429 & 15.429 & 15.429 \\ 
\hline 
War - Mad & 1.524 & 1.524 & 1.524 \\ 
\hline 
War - Rom & 18.857 & 18.857 & 18.857 \\ 
\hline 
Mad - Rom & 1.524 & 12.857 & 2.190 \\ 
\hline 
\end{tabular} 
\end{center}

Once again, we can verify the conservation of flow along with the 50Gbit/s entry in Stockholm and 50Gbit/s output in Rome. We can also see that the solution does not change much if we compare to the one from the basic exercise. 

In conclusion, the solution is not very different from the one obtained in deterministic programming. This is because the scenarios chosen are close to the mean. Also, the same factor is applied to every link in the graph. We could for example consider a scenario where part of the links increase their capacity while other decrease it. We would then have a very different solution. We can also note that, in order to have a feasible region, the factor $p$ must not be too large. For example, if $p=0.2$ then the problem is not feasible. It can easily be seen with the capacities around London. The total mean capacity around London is $11+21+17 = 49$. So if we take 20\% out of that, we get $0.8(11+21+17) = 39.2$. Which is not enough to route a 40 Gbit/s-flow out of London.