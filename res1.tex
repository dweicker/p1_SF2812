The solution to routing the traffic in the network when minimizing the maximum utility in any link can be seen in Figure 2. The figure shows how much of each type of traffic that is to be sent in each arc. The blue numbers, $k=1$, denotes the flow from Stockholm to Rome and the red numbers, $k=2$, denotes the flow from London to Warsaw. 

\begin{center}
\begin{tikzpicture}[->,thick,shorten >=1pt,auto,node distance=5.4cm,
main node/.style={rectangle,draw,font=\sffamily\Large\bfseries},
plain node/.style={circle,draw,font=\sffamily\Large\bfseries}]
  \node[main node] (1) {Stockholm};
  \node[main node] (2) [below of=1] {Berlin};
  \node[main node] (3) [below right of=2] {Warsaw};
  \node[main node] (4) [below left of=2] {Paris};
  \node[main node] (5) [above left of=4] {London};
  \node[main node] (6) [below right of=4] {Rome};
  \node[main node] (7) [below left of=4] {Madrid};
  \node[plain node] (8)[above =0.5cm of 1] {};
  \node[plain node] (9)[left =0.5cm of 5] {};
  \node[plain node] (10)[right =0.5cm of 3] {};
  \node[plain node] (11)[below =0.5cm of 6] {};

  \path[every node/.style={font=\sffamily\small}]
    (1) edge [bend right=20] node[blue] {1.0} (5)
    	edge [bend right=20] node[blue] {12.0} (4)
     	edge [bend right=25] node[blue] {20.6} (2)
    	edge [bend left=20] node[blue] {16.4} (3)
    	edge [bend left=40] node[red] {9.3} (3)
    (2) edge [bend right=25] node[red] {0.9} (1)
    	edge [bend right=25] node[blue] {3.0} (4)
        edge [bend right=20] node[blue] {2.1} (3)
    	edge [bend left=20] node[red] {15.0} (3)
    	edge node[blue] {15.4} (6)
    (3) edge [bend right=25] node[blue] {18.6} (6)
    	edge node[red] {40} (10)
    (4) edge [bend right=25] node[red] {15.9} (2)
    	edge [bend right=20] node[blue] {16.0} (6)
    	edge [bend left=20] node[red] {0.3} (6)
    	edge node[red] {0.9} (7)
    (5) edge [bend right=10] node[red] {8.4} (1)
    	edge [bend left=20] node[red] {17.0} (4)
    	edge [bend right=20] node[blue] {1.0} (4)
		edge [bend right=20] node[red] {14.6} (7)
	(6) edge [bend right=25] node[red] {0.3} (3)
		edge node[blue] {50} (11)
    (7) edge [bend right=85] node[red] {15.4} (3)
    (8) edge node[blue] {50} (1)
    (9) edge node[red] {40} (5);
\end{tikzpicture}
\end{center}
Figure 2. Result of routing for each type of traffic rounded to the nearest decimal.

Addition to the solution of the variable $x_{i,j,k}$, we also solved for the variable $maxUtility$. The minimum value of utility when routing traffic is $$maxUtility = 0.857$$

To determine whether there is any slack in the network, let us investigate the utility in each link.

\begin{table}[H]
\centering
\caption{Utility for each link}
\label{}
\begin{tabular}{|c|c|} \hline
        Link & Utility				\\ \hline
Sto -- Lon & 85.7\%   \\
Sto -- Par & 85.7\%        \\
Sto -- Ber & 85.7\%     \\
Sto -- War & 85.7\%		\\
Lon -- Par & 85.7\%    \\
Lon -- Mad & 85.7\%       \\
Par -- Ber & 85.7\%    \\
Par -- Mad & 2.8\%     \\
Par -- Rom & 85.7\%    \\
Ber -- War & 65.9\%       \\
Ber -- Rom & 85.7\%     \\
War -- Rom & 85.7\%    \\
Mad -- War & 85.7\%        \\
Mad -- Rom & 0\%      \\ \hline
\end{tabular}
\end{table}

For almost all of the links, the utility is $85.7\%$, which means the traffic is spread evenly. However, there are three links in which there is slack; arcs $(Par,Mad), (Ber,War) \text{ and } (Mad,Rom)$. The connection between Madrid and Rome is not used at all and could be taken away without changing the routing of traffic.

On the other hand, let us now investigate in which links Nett should buy extra capacity in order to create more slack, i.e. decrease the maximum utility. One might think that Nett should invest in all links where the utility is $85.7\%$. Though, this is not true because there are some links where the strain is more severe. In other words, there are some key links in the network which limit the minimum value of the utility.

Along with the solution to the variables in the network problem, we also got \textit{marginal values} for all the capacities in the network respectively. Consider the third constraint in the optimization problem, which controls that the traffic does not exceed the capacity. If we were to move all the variables to the left hand side and keep the constants on the right hand side, the marginal values denote the amount that the objective function change if the right hand side were increased by 1.0. In our case, we simply look at which capacities to increase in order to generate the highest decrease in utility.

There are 5 links, with the same marginal value, which we recommend Nett to invest in extra capacity. The links are
$$
(Sto,Lon), (Sto,Par), (Ber,Par), (War,Mad) \text{ and }(War,Rom)
$$